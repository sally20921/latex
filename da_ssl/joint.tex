\documentclass[conference]{IEEEtran}
\IEEEoverridecommandlockouts
% The preceding line is only needed to identify funding in the first footnote. If that is unneeded, please comment it out.
\usepackage{cite}
\usepackage{amsmath,amssymb,amsfonts}
\usepackage{algorithmic}
\usepackage{graphicx}
\usepackage{textcomp}
\usepackage{xcolor}
\def\BibTeX{{\rm B\kern-.05em{\sc i\kern-.025em b}\kern-.08em
    T\kern-.1667em\lower.7ex\hbox{E}\kern-.125emX}}
\begin{document}

\title{Joint Contrastive Learning for Unsupervised Domain Adaptation}

\author{\IEEEauthorblockN{Seri Lee}
\IEEEauthorblockA{\textit{Computer Science and Engineering} \\
\textit{Seoul National University}\\
Seoul, Republic of Korea\\
sally20921@snu.ac.kr} 
}

\maketitle

\begin{abstract}
Enhancing feature transferability by matching marginal distributions has led to improvements in domain adaptation, although this is at the expense of feature discrimination.
In particular, the ideal joint hypothesis error in the target error upper bound, which was previously considered to be minute, has been found to be significant, impairing its theoretical guarantees.
In this paper, we propose an alternative upper bound on the target error that explicitly considers the joint error to render it more manageable.
With the theoretical analysis, we suggest a joint optimization framework that combines the source and target domains.
Further, we introduce Joint Contrastive Learning to find class-level discriminative features, which is essential for minimizing the joint error.
With a solid theoretical framework, JCL employs contrastive loss to maximize the mutual information between a feature and its label, which is equivalent to maximizing the Jensen-Shannon divergence between conditional distributions.
\end{abstract}

\section{Introduction}
\cite{b1}. The major characteristic of domain adaptation is the dataset shift that precludes small target error when the classifier trained on the source domain is directly applied to the unlabeled target data.
A theoretical analysis provided indicates that the target error is upper bounded by the sum of the source error, the domain discrepancy, and the error of the ideal joint hypothesis, while the last term is often treated as constant in the literature.
Several studies have focused on reducing the discrepancy between the marginal distributions of the domains in feature space.

Although making the source and target feature distributions has led to increased accuracy as it enhances feature transferability, it has been at the expense of feature discriminability.
To illustrate, methods based on marginal distribution alignment in the domain level may disregard class-conditional distributions.
As a result, it is possible that decision boundaries traverse high-density regions of the target domain, rendering the learned classifier vulnerable to misclassification.

To enhance feature discrimnability, recent domain adaptation methods have explored two strategies. The first strategy matches the first-order statistics of conditional distributions.
The second strategy uses pairwise loss or triplet loss to learn discriminative features.





\bibliography{refs_through}
\bibliographystyle{plain}

\end{document}
