\documentclass[conference]{IEEEtran}
\IEEEoverridecommandlockouts
% The preceding line is only needed to identify funding in the first footnote. If that is unneeded, please comment it out.
\usepackage{cite}
\usepackage{amsmath,amssymb,amsfonts}
\usepackage{algorithmic}
\usepackage{graphicx}
\usepackage{textcomp}
\usepackage{xcolor}
\def\BibTeX{{\rm B\kern-.05em{\sc i\kern-.025em b}\kern-.08em
    T\kern-.1667em\lower.7ex\hbox{E}\kern-.125emX}}
\begin{document}

\title{Self-supervised Domain Adaptation for Computer Vision Tasks}

\author{\IEEEauthorblockN{Seri Lee}
\IEEEauthorblockA{\textit{Computer Science and Engineering} \\
\textit{Seoul National University}\\
Seoul, Republic of Korea\\
sally20921@snu.ac.kr}
ß
}

\maketitle

\begin{abstract}
Recent progress of self-supervised visual representation learning
has achieved remarkable success on many challenging computer vision
benchmarks. 
However, whether these techniques can be used for domain Adaptation
has not been explored. 
In this work, we propose a generic method for self-supervised domain adaptation,
using object recognition and semantic segmentation of urban scenes as use cases.
Focusing on simple pretext/auxiliary tasks (e.g. image rotation prediction),
we assess different learning strategies to improve domain adaptation effectiveness 
by self-supervision. 
Additionally, we propose two complementary strategies to further boost 
the domain adaptation accuracy on semantic segmentation within our method,
consisting of prediction layer alignment and batch normalization calibration.
The experimental results show adaptation levels comparable to most studied domain
adaptation methods, thus, bringing self-supervision as a new alternative for 
reaching domain adaptation. 
\end{abstract}

\begin{IEEEkeywords}
Domain Adaptation, semantic segmentation, object recognition
\end{IEEEkeywords}

\section{Introduction}
\cite{b1}.

\bibliography{refs_da_ssl}
\bibliographystyle{plain}


\end{document}