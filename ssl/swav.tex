\documentclass[conference]{IEEEtran}
\IEEEoverridecommandlockouts
% The preceding line is only needed to identify funding in the first footnote. If that is unneeded, please comment it out.
\usepackage{cite}
\usepackage{amsmath,amssymb,amsfonts}
\usepackage{algorithmic}
\usepackage[ruled,vlined]{algorithm2e}
\usepackage{graphicx}
\usepackage{textcomp}
\usepackage{xcolor}
\usepackage{verbatim}
\def\BibTeX{{\rm B\kern-.05em{\sc i\kern-.025em b}\kern-.08em
    T\kern-.1667em\lower.7ex\hbox{E}\kern-.125emX}}
\begin{document}

\title{Unsupervised Learning of Visual Features by Contrasting Cluster Assignments}

\author{\IEEEauthorblockN{Seri Lee}
\IEEEauthorblockA{\textit{Computer Science and Engineering} \\
\textit{Seoul National University}\\
Seoul, Republic of Korea\\
sally20921@snu.ac.kr} 
}

\maketitle

\begin{abstract}
Unsupervised image representations have significantly reduced the gap with supervised pretraining, notably with the recent achievements of contrastive learning methods.
These contrastive methods typically work online and rely on a large number of explicit pairwise feature comparisons, which is computationally challenging.
In this paper, we propose an online algorithm, SwAV, that takes advantage of contrastive methods without requiring to compute pairwise comparisons. 
Specifically, our method simultaneously clusters the data while enforcing consistency between cluster assignments produced for different augmentations (or "views") of the same image, instead of comparing features directly as in contrastive learning.
Simply put, we use a "swapped" prediction mechanism where we predict the code of a view from the representation of another view.
Our method can be trained with large and small batches and can scale to unlimited amounts of data. 
Compared to previous contrastive methods, our method is more memory efficient since it does not require a large memory bank or a special momentum network.
In addition, we also propose a new data augmentation strategy, multi-crop, that uses a mix of views with different resolutions in place of two full-resolution views, without increasing the memory or compute requirements.
We validate our findings by achieving 75.3\% top-1 accuracy on ImageNet with ResNet-50, as well as surpassing supervised pretraining on all the considered transfer tasks.
\end{abstract}

\section{Introduction}
\cite{b1}.

Unsupervised visual representation learning, or self-supervised learning, aims at obtaining features without using manual annotations and is rapidly closing the performance gap with supervised pretraining in computer vision.
Many recent state-of-the-art methods build upon the instance discrimination task that considers each image of the dataset (or ``instance'') and its transformations as a separate class.
This task yields representations that are able to discriminate between different images, while achieving some invariance to image transformations.
Recent self-supervised methods that use instance discrimination rely on a combination of two elements: (1) a contrastive loss and (2) a set of image transformations.
The contrastive loss removes the notion of instance classes by directly comparing image features while the image transformations define the invariances encoded in the features.
Both elements are essential to the quality of the resulting networks and our work improves upon both the objective function and the transformations.

The contrastive loss explicitly compares pairs of image representations to push away representations from different images while pulling together those from transformations, or views, of the same image.
Since computing all the pairwise comparisons on a large dataset is not practical, most implementations approximate the loss by reducing the number of comparisons to random subsets of images during training.
An alternative to approximate the loss is the approximate the task-that is to relax the instance discrimination problem. 
For example, clustering

\bibliography{refs_contrastive}
\bibliographystyle{plain}

\end{document}