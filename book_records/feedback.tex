\documentclass[ebook,12pt,oneside,openany]{memoir}
\usepackage[utf8x]{inputenc}
\usepackage[english]{babel}
\usepackage{url}

\begin{document}

\title{On giving feedback}
\maketitle
\section{Lie About Work - People need feedback}

There is an established consensus that \textit{people need feedback},
and that the best companies and the most effective team leaders must figure
out how to give it to them.

Add to this the wonky logic that since success is achieved 
only through \textit{hard work}, and since giving negative feedback,
receiving negative feedback, and fixing mistakes are all \textit{hard work},
therefore negative feedback causes success, and you can begin to see why 
our faith in feedback, and specifically negative feedback, is so firmly rooted - why
we ``know for sure'' that feedback is helpful and that our colleagues need it.
 
But this just ain't so.

For all of Snapchat's early users this was a relief. The very absence
of permanent feedback allowed them to be more casual, more at ease,
and more real, and this safe, attentive place attracted them in the millions.

If the Snapchat example is any guide, it would seem that at root, social media 
is more about publishing - about positive self-presentation.
It matters less to us whether this ``self'' is truly us, or whether, 
as many have observed, our online selves are aspirational projections,
than it matters to us that others see us, and like us. 
We aren't looking for feedback. We're looking for an audience, and all of us -
not just millennials-seem drawn to places that provide us with a way to meet our audience
and gain its approval.
What we want from social media is not really feedback. It's attention, and the lesson
from the last decade is that social media is an attention economy-some users seeking it, 
some supplying it-not a feedback economy.

The Snapchat growth story is only the most recent addition to a large 
body of evidence about the human need for uncritical attention.
More recently, epidemiologists, psychometricians, and statisticians have shown
that by far the best predictor of heart disease, depression, and suicide is loneliness-
if you deprive us of the attention of others, we wither.

It took a while to figure out what was going on, but the consensus that 
ultimately emerged from the Hawthorne experiments has had a profound effect on the 
science of work. 
The conclusion was not that workers craved a brighter workplace or a tidier
one, or for that matter, a darker one or a messier one.
Instead, what the workers were responding to was attention.
Each of these interventions demonstrated to the workers that management was interested in them
and their experience, and they liked that. 

The truth is, then, is that \textit{people need attention}-and when you give it to us
in a safe and nonjudgmental environment, we will come and stay and play and work.

But it's a bit more complicated than that, as it turns out, because feedback-even negative feedback-
is still attention. And it's possible to quantify the impact of negative attention, 
if you will, versus positive attention, versus no attention at all, and thereby better understand what sort of attention
we most want at work.

To create pervasive disengagement, ignore your people. If you pay them no attention
whatsoever-no positive feedback; no negative feedback; nothing- your team's engagement will plummet.

They found that negative feedback is forty times more effective, as a team leadership approach, than ignoring people.
For those employees whose leaders' attention was focused on fixing their shortcomings, 
the ratio of engaged to disengaged was two to one.
if we recall that most of us experience mainly negative feedback in our professional lives;
and if we consider what the researchers found when they looked at positive attention, 
then this ratio of two to one becomes much more worrying. 

Because the third finding was this: for those employees given mainly positive attention-that is,
attention to what they did best, and what was working most powerfully for them-the ratio
of engaged to disengaged rose to sixty to one.

Positive attention, in other words, is \textit{thirty times} more powerful than negative attention 
in creating high performance on a team. 

So while we may occasionally have to help people get better at something that's
holding them back, if paying attention to what people \textit{can't} do is our default setting as team leaders,
and if all our efforts are directed at giving and receiving negative feedback more often and more efficiently,
then we're leaving enormous potential on the table.
\textbf{People don't need feedback.}
They need attention, and moreover, attention to what they do best. 

But what about learning? If all we get is attention to our strengths, how will we ever develop?
Again, our informal theories of work-our ``know for sure'' theories-let us down.
We seem to accept, on its face, the idea that ``strengths'' go at one end of the scale
and ``areas for improvement'' or ``areas of opportunity'' go at the other, that areas of high performance
are where we are most complete and areas of low performance are where we should, and can, grow.

But as we saw in the last chapter, the single most powerful predictor of both team performance and team engagement is the sense
that ``I have the chance to use my strengths every day at work''.
Now, we tend to think of \textbf{performance} and \textbf{development}
as two separate things, as though development or growth is something that exists outside of the present-day work.
But development means nothing more than doing our work a little better each day, so increasing performance and creating
growth are the same thing. A focus on strengths increases peformance.
Therefore, a focus on strengths is what creates growth.

The best team leaders seem to know this. They reject the idea that the most important focus of their time is 
people's shortcomings, realizing instead that, in the real world, each person's strengths are in fact her areas of great opportunity
for learning and growth;and that consequently, time and attention devoted to contributing to these strengths intelligently will yield
exponential return now and in the future. 

At the microscopic level learning appears to be a function of neurogenesis:
the growth of new neurons. And, as many recent studies have shown, the brain-though it goes through 
its most frenzied periods of synapse growth and synapse pruning during childhood and adolescence-
never loses its ability to create more neurons and more synaptic connections between those neurons.
What the brain science also reveals is that while the brain does continue to grow throughout life, 
each brain grows differently. 
Because of your genetic inheritance and the oddities of your early childhood environment, 
your brain's wiring is utterly unique-no one has ever had a brain wired just like yours, and given the brain's complexity,
no one ever will.
Some parts of your brain have tight thickets of synaptic connections, while other parts are far less dense.
And when we examine your brain's growth-when we count the new neurons and their connections-
it turns out that you grow far more neurons and synaptic connections where you already have the most 
preexisting neurons and synaptic connections.
Perhaps this is caused by nature's harshly efficient use-it-or-lose-it design, or perhaps, with so much preexisting
biological infrastructure supporting your densest synaptic regions, it is simply easier to forge new connections where you
already have lots. Either way, we now know that, though every brain grows, 
each grows most where it's already strongest.

So the weight of the neurological evidence supports the notion that your strengths \textit{are}
your developmental areas-that these are, biologically speaking, one and the same. 
Neurological science can also tell us what happens in response to a deliberate focus on strengths 
instead of weaknesses.

In the brains of the students who received negative feedback, the sympathetic nervous system lit up.
This is the ``fight or flight'' system, the system that mutes the other parts of the brain and thus
allows us to focus only on the information most necessary to survive.
When this part of the nervous system is triggered, your heart rate goes up, endorphins flood your body,
your cortisol levels rise, and you tense for action.
This is your brain on negative feedback: it responds as if to a threat, and it narrows its activity.
The strong negative emotions produced by criticism inhibits access to existing neural circuits and invokes 
cognitive, emotional and perceptual impairment.

Negative feedback doesn't enable learning. It systematically inhibits it and is, 
neurologically speaking, how to create \textit{impairment}.

In the students who received attention focused on their dreams and how they might 
go about achieving them, however, the sympathetic nervous system was not activated.
Instead it was the parasympathetic nervous system that lit up.
This is sometimes referred to as the ``rest and digest'' system.
The parasympathetic nervous system stimulates adult neurogenesis, a sense of well being, 
better immune system functioning, and cognitive, emotional and perceptual openness.

In other words, positive, future-focused attention gives your brain access to more regions of itself
and thus sets you up for greater learning. \textbf{We're often told that the key to learning is to get out of our comfort zones,
but this finding gives the lie to that particular chestnut-take us out of our comfort zones and our brains stop 
paying attention to anything other than surviving the experience.}
\textbf{It's clear that we learn most in our comfort zone, where our neural pathways are most concentrated.}
It's where we're most open to possibility, and it's where we are most creative and insightful.

If you want your people to learn more, pay attention to what's working for them right now, and then build on that.

The question is, how? There's one thing you can start to do immediately: 
get into the conscious habit of looking for what's going well for each of your team members.
The pull to look at the negative is a very strong one-the brain is like Velcro for negative experiences, 
but Teflon for positive ones-which is why making this a conscious habit is so important.

You can fix a machine, you can fix a process, but you can't fix a person in the same way.

IF what you want is improvement, then it should be whenever someone on your team does something that really works.
The goal is to consciously spend your days alert for those times when someone on your team does something so easily and 
effectively that it rocks you and then find a way of telling that person what you just saw.

The best way to define and know the right way was to look at those plays where the player had done it right.
So he set about capturing these distinctive moments of excellence and offering them up to each player. 
``We only replay your winning plays''

His instincts told him that each person would learn best how to improve his performance 
if he could see, in slow motion, what his own personal versions of excellence looked like. 
Really great performance often happens in a state of flow, such that we're barely conscious of what we're doing.

In so doing, he hoped not only that they would feel more confident, but also that they'd be in a better position to 
repeat and build on their unique strengths in action. 

You can do this, too. Nowadays \textit{recognition} has become a synonym for \textit{praise},
but in doing so has moved some way from its origins.
To re-cognize a person, in essence, means to come to know him anew.
Recognition, in its deepest sense, is to spot something valuable in a person and 
then to ask her about it, in an ongoing effort to learn who she is when she is at her best. 

What you'll want to do is tell the person what you experienced when that moment of excellence caught your attention-
your instantaneous reaction to what worked. 
For a team member, nothing is more believable and thus more powerful, than your sharing what you saw from her 
and how it made you feel.
These are your reactions, and when you share them with specificity and with detail, you aren't judging her or 
rating her or fixing her. 

Remember that if a team member screws something up, you're merely remediating-
and that remediating what's wrong, so a mistake won't happen again, moves you no closer to creating excellence performance.
Excellence is not the opposite of failure: \textbf{we can never create excellent performances by only fixing poor ones}.
Mistake fixing is just a tool to prevent failure.

To conjure excellence from your team requires a different focus for your attention.

Tell me what I'm doing wrong, he'll say. Try to resist the powerful temptation to jump in 
with your very best advice. The advice given to you by a leader who is not you will not necessarily work for you.
The best team leaders understand that the path you will take to your best performance will be unpredictably different 
from theirs. 

When a team member comes to you asking for advice, then, start with the present.
Ask your colleague to tell you three things that are working for him right now.
In doing that, you're priming his mind with oxytocin-which here is better thought of as the creativity drug.

Next, go to the past. Ask him when he had a problem like this in the past, what he did that worked.
It's highly likely that he has encountered this problem before and found himself similarly stuck.



\end{document}
