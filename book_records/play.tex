\documentclass[ebook,12pt,oneside,openany]{memoir}
\usepackage[utf8x]{inputenc}
\usepackage[english]{babel}
\usepackage{url}

\begin{document}

\title{On Play}
\maketitle
\section{Why Play?}
Researchers from every point of the scientific compass now know that 
play is a profound biological process.
It has evolved over eons in many animal species to promote survival.
It shapes the brain and makes animals smarter and more adaptable.
In higher animals, it fosters empathy and makes possible complex social groups.
For us, play lies at the core of creativity and innovation.

Of all animal species, humans are the biggest players of all. 
We are built to play and built through play. 
When we play, we are engaged in the purest expression of our humanity, the truest expression of our individuality.
Is it any wonder that often the times we feel most alive, those that make up our best memories, are moments of play?

I have used play therapies to help people who are clinically depressed.
I have found that remembering what play is all about and making it part of our daily lives
are probably the most important factors in being a fulfilled human being.
The ability to play is critical not only to being happy, but also to sustaining social relationships and being a creative,
innovative person.

At some point as we get older, however, we are made to feel guilty for playing.
We are told that it is unproductive, a waste of time, even sinful.
The play that remains is mostly very organized, rigid, and competitive.
We strive to always be productive, and if an activity doesn't teach us a skill, make us money, or get on the boss's good side,
then we feel we should not be doing it.
Sometimes the sheer demands of daily living seem to rob us of the ability to play.

The beneficial effects of getting just a little true play can spread through our lives, actually making us more productive and happier in everything we do.

There is a kind of magic in play. 
What might seem like a frivolous or even childish pursuit is ultimately beneficial.
It's paradoxical that a little bit of nonproductive activity can make one enormously more productive and invigorated in other aspects of life.

Once people understand what play does for them, they can learn to bring a sense of excitement and adventure back to their lives,
make work an extension of their play lives, and engage fully with the world.

This book is about learning to harness a force that has been built into us through millions of years of evolution,
a force that allows us to both discover our most essential selves and enlarge our world.
We are designed to find fulfillment and creative growth through play.

\section{What is play?}
Another reason I resist defining play is that at its most basic level,
play is a very primal activity. 
It arises out of ancient biological structures that existed before our consciousness or our ability to speak.
In us, play can also happen without a conscious decision. 
Like digest and sleep, play in its most basic form proceeds without a complex intellectual framework.

\subsection{Properties of Play}
\begin{itemize}
    \item purposeless
    \item voluntary
    \item inherent attraction
    \item freedom from time
    \item diminished conscious of self
    \item improvisational potential
    \item continuation desire
\end{itemize}

\section{Why do we play?}
I started to see that play is a tremendously powerful force throughout nature.
In the end, it is largely responsible for our existence as sentient, intelligent creatures.

Why do animals play? In a world continuously presenting unique challenges and ambiguity, play prepares these bears for an evolving planet.
Play is incredibly pervasive in the animal kingdom. 

Why would they waste time and energy in nonproductive activity like play? 
They found that the bears that played the most were the ones who survived the best.
This is true despite the fact that playing takes away time, attention, and energy from activities like eating,
which seem at first glance to contribute more to the bears' survival.

\section{The brain on play}
Animals that play a lot quickly learn how to navigate their world and adapt to it.
In short, they are smarter. A researcher reported that there is a strong positive link
between brain size and playfulness for mammals in general.
Another renowned researcher has shown that active play selectively stimulates brain-derived neurotrophic faactor in the amygdala
and the dorsolateral prefrontal cortex. The amount of play is correlated to the development of the brain's frontal cortex, which is 
the important brain region responsible for much of what we call cognition.

Byers speculates that during play, the brain is making sense of itself through simulation and testing.
Play activity is actually helping sculpt the brain. 
In play we can imagine and experience situations we have never encountered before and learn from them.
We can learn lessons and skills without being directly at risk. 

Gerald Edelman has created a theory about how new information is functionally integrated into the brain.
Edelman describes how our perceptual experiences are coded within the brain in scattered maps, eaach of which is a complex network
of interconnected neurons.
The vitality of these maps depend on the active and incessant orchestration of countless details.
It seems likely that this orchestration happens most fully through play.

Just as in children, adult streams of consciousness are enriched through the simulations of childlike imaginative play.
We all daydream about events in our future. These thoughts leave an imprint on our brains.

The truth is that play seems to be one of the most advanced methods 
nature has invented to allow a complex brain to create itself.

\section{The drive to play}
Play seems to be so important to our development and survival that the impulse to play has become a biological drive.
The brain can keep developing long after we leave adolescence and play promotes that growth.
We are designed to be lifelong players, built to benefit from play at any age.
The human animal is shaped by evolution to be the most flexible of all animals: as we play,
we continue to change and adapt into old age. 

Not only do we look more like chimpanzee infants than adults, we act more like them.
This quality of retained immaturity goes deeper than our round faces and essentially hairless bodies.
There is a trade-off in staying young. Neoteny tends to be more flexible but vulnerable, while maturity is stronger but more rigid or brittle.

We are designed by nature and evolution to continue playing throughout life.
Lifelong play is central to our continued well-being, adaptation, and social cohesiveness.
Neoteny has fostered civilizations, the arts, and music. 
While neoteny has its drawbacks, it's simply how we are built.
Play very likely continues to prompt continued neurogenesis throughout our long lives.

\section{Play in adulthood}
How do adults play? The answer is not as obvious as it might seem. 
Many of the things we regard as play may have the qualities of work.
And what to many people might seem like work may really be built on a foundation of play.

These emotions don't allow them to feel the playful, out-of-time, in-the-zone, doing-it-for-its-own-sake sensation that 
accompanies joyful playfulness.

It really depends on the emotions experienced by the runner.
Play is a state of mind, rather than an activity. 
It is also self-motivating and makes you want to do it again. 
We have to put ourselves in the proper emotional state in order to play.

The work that we find most fulfilling is almost always a recreation and extension
of youthful play. 
Over the years, I've observed that people have a dominant mode of play that falls into one of eight types. 

\section{living the playful life}
The opposite of play is not work. The opposite of play is depression.
Our inherent need for variety and challenge can be buried by an overwhelming sense of responsibility.
Over the long haul, when these spice-of-life elements are missing, what is left is a dulled soul.

The quality that work and play have in common is creativity. 
Respecting our biologically programmed need for play can transform work.
It can bring back excitement and newness to the job. 
True play that comes from our own inner needs and desires is the only path to finding lasting joy
and satisfaction in our work. In the long run, work does not work without play.

Play gives people the emotional distance to rally. Sometimes when a situation is really heading south, a moment of 
imaginative play is the only thing that provides enough distance to see the way out of a predicament.

By far the biggest reason that companies want to talk to me about play is its role in creativity and innovation.
Those who study creativity find that the process is by nature contradictory and paradoxical, which is why it can seem so mysterious.
On an individual level, your creativity also needs to be protected, not only from outside critics, but also from your own internal critic. 
Allow yourself to be abundant in your creativity, at first not making judgements about what you think, feel or do.
Simply play with your ideas, with how you do things.

There is a great deal of evidence that the road to mastery of any subject is guided by play.
To become a master, the pupil has to go beyond what is known, has to learn what has not been shown by others in the field.
Finally, work that is devoid of play is either boring or a grind.
We can get pretty far through sheer will-power, and some people have prodigious powers of perfectionism, self-denial, and suffering.
Ultimately, though, people cannot succeed in rising to the highest levels of their field if they don't enjoy what they are doing, if they don't
make time for play. 
Having a fierce dedication to grinding out the work is often not enough.
Without some sense of fun or play, people usually can't make themselves stick to any discipline long enough to master it.

People reach the highest levels of a discipline because they are driven by love, by fun, by play.

If play is so necessary to our work, why do we lose it? The answer is that we are both 
pushed and pulled away from play. 
As we get serious about career, get married, have a family, move up the ladder at work, we feel we are 
inexorably pulled away from any time for personal play.
If anyone goes without play for too long, grinding out the work that is expected of them, they will
at some point look at their lives and ask, ``Is this all there is?''
The acquisition of good grades or a big bonus, if not connected to the heart of life, is dispiriting.
In addition to being pulled away from play, we are pushed from play, shamed into rejecting it by a culture
that doesn't understand the human need for it and doesn't respect it.

Most of the time, we have so internalized society's messages about play being a waste of time that 
we shame ourselves into giving up play. 
Joy is our birthright, and is intrinsic to our essential design. 

Enjoying your work too much might be taken as a sign that you are not working hard enough or don't have enough to do.
When we lack that feeling of lightness in what we do it should be taken as a warning sign.

\end{document}