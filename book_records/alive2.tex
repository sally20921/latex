\documentclass[ebook,12pt,oneside,openany]{memoir}
\usepackage[utf8x]{inputenc}
\usepackage[english]{babel}
\usepackage{url}

\begin{document}

\title{Alive At Work}
\maketitle
\section{The Seeking System}
A majority of employees don't feel they can be their best selves at work.
They don't feel they can leverage their unique skills or find a sense of purpose in what they do.

Here's the thing: many organizations are deactivating the part of employees' brains called the \textit{seeking system}.
Our seeking systems create the natural impulse to \textbf{explore} our worlds, \textbf{learn} about our environments, and extract meaning from our circumstances.
When we follow the urges of our seeking system, it release dopamine-a neurotransmitter linked to motivation and pleasure-that makes us want to explore more.

When our seeking system is activated, we feel \textbf{motivated, purposeful, and zestful}. \textbf{We feel more alive}.

\textbf{Exploring, experimenting, learning}: this is the way we're designed to live. Organizations were purposely designed to suppress our natural impulses to \textbf{learn} and \textbf{explore}.

When people work under these conditions, they become \textbf{cautious, anxious, and wary}. They start to experience depressive symtoms.
They begin to believe that their current state is unchangeable, and they disengage from work. 

With their \textbf{explorable worlds and open-ended outcomes}, games are catnip for our brains.
She became hooked because it allowed her to \textbf{express} her unique skills as part of a team, \textbf{explore new things}, and find a sense of purpose-the three triggers that activate the seeking system.

\textbf{Self-expression, experimentation, and a sense of purpose}: these are the switches that light up our seeking systems.

One of the most basic emotional systems pertains to a functionally identifiable neural circuit that depends on dopamine, and that emotional system might be called \textbf{interest, anticipation, or seeking}.
These circuits appear to be major contributors to our feelings of engagement and excitement as we seek the material resources needed for bodily survival, and also when we pursue the cognitive interests that bring positive existential meaning into our lives.
This happens when we are \textbf{curious} or \textbf{find something unexpected}, \textbf{or anticipate something new}.
As a result, we experience a jolt of dopamine, which feels pleasurable and can be \textbf{thrilling}.
And \textbf{since dopamine regulates our perception of time, we experience time differently}.

When the seeking system is activated, we experience persistent feelings of \textbf{interest}, \textbf{curiosity}, \textbf{sensation seeking}, and in the presence of a sufficiently complex cortex, the search for higher meaning.
Zest leads people to live life with a sense of \textbf{excitement, anticipation, and energy}.
We see life or work as an \textbf{adventure}. We approach new situations and changes with \textbf{enthusiasm and excitement} instead of \textbf{apprehension and anxiety}.

\textbf{When we're excited and follow our body's intrinsic urge to learn new things, we become more creative and productive}.
When singers interpret their arousal as excitement, their seeking systems surge, causing them to be more \textbf{playful}, \textbf{optimistic}, and \textbf{creative}.

Positive emotions improve problem solving because people are better able to marshal their cognitive resources to cope with the task at hand, instead of being encumbered by fear and threat.
When people try to become calm under physiological arousal, they are telling themselves that the arousal is bad.
They code the same arousal as \textbf{threat}, \textbf{anxiety}, which activates \textbf{fear}.
When athletes interpret their high arousal as \textbf{excitement}, they are more likely to exhibit \textbf{playful, learning-oriented behaviors}.

When you increase \textbf{enthusiasm} and \textbf{excitement}, you improve problem solving and creativity.
\textbf{This is how most people want to feel in their jobs}.

It shows that the seeking system prompts an \textbf{intrinsic urge to explore}, rather than giving an \textbf{extrinsic reward for an action}.
When the seeking system is stimulated, an animal feels the \textbf{urge to explore and investigate}, to find whatever is potentially useful in the environment.

The seeking system doesn't reward us for innovation and creativity, but rather it drives and propels those behaviors.
This animating effect of the seeking system is optimal in work settings because it \textbf{urges us into action} instead of making us \textbf{complacent}.

Through evolution, we've retained our emotional impulses to \textbf{explore, experiment, and learn}.

The seeking system offers a biological way to understand and predict \textbf{exciting and purposeful activities and feelings outside the realm that is normally thought of as rewarding in a pleasurble or hedonic sense}.
The science suggests that this is important when it comes to cellular functioning and health.

Those experienced dogs had learned \textbf{helplessness}. They had learned to give up and \textbf{passively accept the shock}.
Learned helplessness alters our emotional states, lowers our motivation, and changes our cognitive reasoning.
Employees begin to ignore the urges of their seeking system. They shut off the dopamine and let their \textbf{anxieties dominate}.
An attitude that results in disengagement at work and depressive symptoms in life. They end up making a living but not life.

But say there was a battle between our seeking systems and our fear systems. Which one would win? For evolutionary reasons, fear will win.
To the best of our scientific knowledge, the basic urge to play exists among most mammals. 
Play is how we learn what we're capable of. 

The smell of the fur activated rats' innate fear system, and their play was completely inhibited.
What we are seeing here is the inhibiting relationship between the seeking system and the fear system.
In all species that have been studied, \textbf{playfulness} is inhibited by negative emotions such as fear.

Fear-based management solved the problem of how to control so many employees when you couldn't develop trust.
This became the root of employees' learned helplessness at work.
The seeking system is built for this \textbf{proactive approach}, because it creates \textbf{enthusiasm and curiosity}.
Fine-tuned control that exploits existing processes makes it hard for workers to \textbf{explore} and \textbf{experiment} with new alternatives.
By definition, the outcomes of \textbf{experimenting and playing are uncertain}.
Policies that make us anxious about losing pay, promotions, and status act like the cat fur: our fear system shuts off our seeking system.
Biologically, this inhibits our creativity and \textbf{desire to play}.
\textbf{We just don't ever feel like there is enough time to explore if our behaviors and outcomes are all tightly mapped out in advance, with financial and career implications if we miss them}.
The frame of scientific management leaves our seeking system very little chance.


\end{document}