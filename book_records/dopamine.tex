\documentclass[ebook,12pt,oneside,openany]{memoir}
\usepackage[utf8x]{inputenc}
\usepackage[english]{babel}
\usepackage{url}

\begin{document}

\title{On Dopamine}
\maketitle
\section{The Molecule of More}
\subsection{Introduction}
In your brain the down world is managed by a handful of chemicals-neurotransmitters that let you experience
satisfaction and enjoy whatever you have in the here and now.
But when you turn your attention to the world of up, your brain relies on a different chemical-a single molecule-
that not only allows you to move beyond the realm of what's at your fingertips, but also motivates you to pursue, to control,
and to possess the world beyond your immediate grasp.
It drives you to seek out those things far away, both physical things and things you cannot see,
such as knowledge, love, and power. 

Those down chemicals allow you to experience what's in front of you. 
They enable you to savor and enjoy, or perhaps to fight or run away, right now.
The up chemical is different. It makes you desire what you don't yet have,
and drives you to seek new things.
It rewards you when you obey it, and makes you suffer when you don't.
It is the source of creativity and further along the spectrum, madness.
It is the key to addiction and the path to recovery.
It is the bit of biology that makes an ambitious executive sacrifice everything in pursuit of success,
that makes successful actors and entrepreneurs and artists keep working long after they have all the money and fame.
and that makes a satisfied husband or wife risk everything for the thrill of someone else.
It is the source of the undeniable itch that drives scientists to find explanations and philosophers to find order, reason, and meaning.
It is why we seek and succeed. It is why we discover and prosper. It is also why we are never happy for very long.

To your brain, this single molecule is the ultimate multipurpose device, urging us, through thousands of neurochemical processes, 
to move beyond the pleasure of just being, into exploring the universe of possibilities that come when we imagine.
Mammals, reptiles, birds, and fish all have this chemical inside their brains, but no creature has more of it than a human being.

This is dopamine, and it narrates no less than the story of human behavior.

\subsection{Love}
Dopamine, they discovered, isn't about pleasure at all. Dopamine delivers a feeling much more influential.
Understanding dopamine turns out to be the key to explaining and even predicting behaviors across a spectacular range of human endeavors:
creating art; seeking success; and falling in love.
From that, a new hypothesis arose: dopamine activity is not a marker of pleasure. 
It is a reaction to the unexpected-to possibility and anticipation.

As human beings, we get a dopamine rush from similar, promising surprises.
But when these things become regular events, their novelty fades, and so does the dopamine rush.

Our brains are programmed to crave the unexpected and thus to look to the future, where every exciting possibility begins.
But when anything, including love, becomes familiar, that excitement slips away, and new things draw our attention.

That happy error is what launches dopamine into action. It's the thrill of the unexpected good news.

In fact, the mere possibility of a reward prediction error is enough for dopamine to swing into action.
It's the pleasure of anticipation-the possibility of something unfamiliar and better.

Yet sometimes when we get the things we want, it's not as pleasant as we expect.
Dopaminergic excitement doesn't last forever, because eventually the future becomes the present.

Passion rise when we dream of a world of possibility, and fades when we are confronted by reality.

This is the defining characteristic of things in the extrapersonal space: to get them requires effort, time, and in many cases, planning.
By contrast, anything in the peripersonal space can be experienced in the here and now.
Those experiences are immediate. 

The brain works one way in the peripersonal space and another way in the extrapersonal space.
The division is so fundamental that separate pathways and chemicals evolved in the brain to handle peripersonal and extrapersonal space.
Things in the distance, things we don't have yet, cannot be used or consumed, only desired.
Dopamine has a very specific job: maximizing resources that will be available to use in the future; the pursuit of better things.

Every part of living is divided in this way: we have one way of dealing with what we want, and another way of dealing with what we have.
Love must shift from something we anticipate to something we have to take care of.
These are vastly very different skills, which is why over time the nature of love has to change-and why love fades away at the end of the dopamine 
thrill we call romance.

There's a dark side to dopamine. Now the pigeon never knew when the food would come.
Every reward was unexpected. The birds became excited. They pecked faster. Something was spurring them on to greater efforts.
Dopamine, the molecule of surprise, had been harnessed.

The novelty that triggers dopamine doesn't go on forever. From dopamine's point of view, having things is uninteresting.
It's only getting things that matters.
Dopamine has no standard for good, and seeks no finish line.
The dopamine circuits in the brain can be stimulated by only by the possibility of whatever is shiny and new, 
never mind how perfect things are at the moment. The dopamine motto is more.

Dopamine isn't a pleasure molecule, after all. It's the anticipation molecule.
To enjoy the things we have, as opposed to the things that are only possible, 
our brains must transition from future-oriented dopamine to present-oriented chemicals,
a collection of neurotransmitters we call the HNs.
They include serotonin, oxytocin, endorphins.
As opposed to the pleasure of anticipation via dopamine, 
these chemicals give us pleasure from sensation and emotion.
When the HNs take over in the second stage of love, dopamine is suppressed. 
In fact, though dopamine and HN circuits can work together, 
under most circumstances, they counter each other.

\subsection{Dominance}
The nuerotransmitter dopamine is the source of desire (via the desire circuit) and tenacity (via the control circuit).
The passion that points the way and the willpower that gets us there.
Usually the two work together, but when desire fixates on things that will bring us harm in the long run, dopaminergic willpower
turns around and does battle with its companion circuit.

Willpower isn't the only tool control dopamine has in its arsenal when it needs to oppose desire.
It can also use planning, strategy, and abstraction, such as the ability to imagine the long-term consequences of alternate choice.
But when we need to resist harmful urges, willpower is the tool we reach for first.
As it turns out, that might not be such a good idea. 

Willpower is like a muscle. It becomes fatigued with use, and after a fairly short period of time, it gives out.
Even though it is possible to strengthen willpower, it's still now the answer to long-term, enduring change.

The goal of addiction psychotherapy is to pit one part of the brain against another.
Part of the dopamine desire circuit becomes malignant in drug addiction, pushing the addict
into compulsive, uncontrollable use. It has to be opposed by an equally potent force.

Motivational Enhancement Therapy: Desire Dopamine versus Desire Dopamine.
Part of them wants nothing more than to use drugs, but there are other, weaker desires as well.
Those desires can be strengthened. 
Desire not only gives us motivation to act; it also gives us patience to endure.
In MET, patients tolerate feeling resentful and deprived, the punishment of disappointed dopamine, 
because they know it will lead to something better.

Cognitive Behavioral Therapy: Control Dopamine Versus Desire Dopamine.
CBT uses the planning ability of control dopamine to defeat the raw power of desire dopamine.
Alchoholics in CBT learn to arm themselves against cue-triggered craving in a number of different ways.

Twelve-Step Facilitation Therapy: HN Versus Desire Dopamine.
The dopamine system as a whole evolved to maximize future resources. 
In addition to desire and motivation, we also possess a more sophisticated circuit that gives us the ability to 
think long term, make plans, and use abstract concepts such as math, reason, and logic.
Looking into the longer-term future also gives us the tenacity we need to overcome challenges and accomplish things that take a long time.
The control circuit suppresses HN emotion, allowing us to think in a cold, rational way that's often required
when hard decisions need to be made.

Dopamine yields not just desire but also domination. It gives us the ability to bend the environment and even other people 
to our will. But dopamine can do more than give us dominion over the world: it can create entirely new worlds, worlds that may be 
so astonishing, they could have been created only by a genius-or a madman. 

\subsection{Creativity and Madness}
Yet madness and genius, the worst and best the brain can do, both depend on dopamine.
Because of this basic chemical connection, madness and genius are more closely connected to each other 
than either is to the way ordinary brains work.

One clue comes from what we know about how to treat schizophrenia.
Psychiatrists prescribe medications called antipsychotics that reduce activity within the dopamine desire circuit.
At first glance, that seems odd. Stimulation of the desire circuit typically leads to excitement, wanting, enthusiasm,
and motivation. 

Salience refers to the degree to which things are important, prominent, or conspicuous.
Things are salient if they have the potential to affect your future. Things are salient if they trigger desire dopamine.
They broadcast the message, Wake up. Pay attention. Get excited. This is important.

Too much salience, or any salience at all at the wrong time, can create delusions.
The triggering event rises from obscurity to importance.
People with schizophrenia control their dopamine activity by taking medications that block dopamine receptors.
Unfortunately, antipsychotic medications block dopamine all over the brain,
and blocking the control circuit in the frontal lobes can make certain aspects of the illness worse, such as difficulty paying attention and reasoning
with abstract concepts.

We often speak this way when we're excited. 
Desire dopamine gets revved up, and overwhelms control dopamine's more logical approach to communication.
Like people with mental illness, creative people will at times experience their thoughts running free.

Material things, objects in the HN peripersonal space, can be experienced with all five senses.
How do we perceive something that is so far away that we can't even see it? We use our imagination.

Models are imaginary representations of the world that we build in order to better understand it.
Models also allow us to abstract, to take specific experiences and use them to craft abroad, general rules.
Using imagination, we project ourselves into various possible futures, mentally experience them, then decide 
how we're going to get the most out of what we see. 

Mental time travel is a powerful tool of the dopamine system. 
Models are powerful tools, but they have disadvantages. They can lock us into a particular way of thinking,
causing us to miss out on opportunities to improve the world.

He found that when they discovered solution to the problem, the front of their brains on the right side was activated.
He wondered if this part of the brain was also involved in model breaking.
Maybe it's because when we are being creative, we behave a little bit like a person with schizophrenia.
We stop inhibiting aspects of reality that we had previously written off as unimportant, and we attach salience
to things we once thought were irrelevant.

Analogies represent a very dopaminergic way of thinking about the world.
An analogy pulls out the abstract, unseen essence of a concept, and matches it with a similar essence of an apparently unrelated concept.
The ability to draw a connection between two things that had previously appeared to be unrelated is an important part of creativity.
Dopaminergic drugs can do the same thing. Others experience enhanced creativity.  

Abstract, detached from the real world of the senses, dreams are dopaminergic.
Dopamine is unleashed during dreaming, freed from the restraining influence of the reality-focused HN neurotransmitters. 
This freedom allows dopamine circuits to generate the bizarre connections that are the hallmarks of dreams.
They found that fantasies produced immediately after dreaming were more elaborate.


The fine arts and the hard sciences have more in common than most people believe, 
because both are driven by dopamine. 
They both require the ability to look beyond the world of senses into a deeper, more profound 
world of abstract ideas. 
Elite societies of scientists are filled with artistic souls.
The better you are at managing the most complex, abstract ideas,
the more likely you are to be an artist.

Music and math go together because elevated levels of dopamine often come as a package deal:
if you are highly dopaminergic in one area, 
you're likely to be highly dopaminergic in others.
Scientists are artists and musicians are mathematicians. 
But there's a downside. Sometimes having lots of dopamine is a liability.

High levels of dopamine suppress HN functioning, so brilliant people are often 
poor at human relationships. 
The abstract concepts of social justice and humanity came easily, but the concrete 
experience of encountering another person was too hard.

Understanding how the brains of geniuses work provides further insight into the dopaminergic 
personality, and the different ways it can manifest itself.
We've already seen the impulsive pleasure-seeker who has difficulty maintaining long-term relationships and 
is vulnerable to addiction. 
We've also seen the detached planner who would rather stay late at the office than enjoy time with friends.
In addition, the dopaminergic genius is so focused on his internal world of ideas that he neglects anything 
having to do with the real world of here and now.

These three personality types appear to be very different on the surface, but they all have something in common.
They're overly focused on maximizing future resources at the expense of appreciating the here and now.
The pleasure seeker always wants more. No matter how much he gets, it's never enough.
No matter how much he looks forward to some promised pleasure, he is incapable of finding satisfaction in it.
As soon as it comes he turns his attention to what's next.
The detached planner also has a future/present imbalance.

Large population studies have also found a genetic component of dopaminergic character.
Based on this evidence, it seems likely that Newton had elevated levels of dopamine that contributed to his brilliance, 
and his psychotic breakdown.

Dopamine gives us the power to create. It allows us to imagine the unreal and connect the seemingly unrelated.
It allows us to build mental models of the world that transcend mere physical description, moving beyond sensory 
impressions to uncover the deeper meaning of what we experience.
Dopamine demolishes its own models so that we can start fresh and find new meaning in what was once familiar.

But that power comes at a cost. The hyperactive dopamine systems of creative geniuses put them at risk of mental illness.
What they care about most is their passion for creation, discovery or enlightenment.
They never relax, never stop to enjoy the good things they have.

\subsection{Politics}
The characteristics the study eventually associated with liberals-risk-taking, sensation-seeking,
impulsivity, and authoritarianism-are the characteristics of elevated dopamine.
But do dopaminergic people really tend to support liberal policies? It seems that the answer is yes.
Progressives are idealists who use dopamine to imagine a world far better than the one we live in today.
Dopaminergic people tend to be creative. 
They also work well with abstract concepts. 
They like to pursue novelty and have a general dissatisfaction with the status quo.

Other studies support a link between a genetic disposition to a dopaminergic personality
and a liberal ideology. While conservatives on average may lack some talents of the dopaminergic left,
they are most likely to enjoy the advantages of a strong HN system.
These include empathy and altruism, and the ability to establish long-term monogamous relationships.

If you're highly dopaminergic, the most important part of sex probably occurs prior to the main event.

\end{document}