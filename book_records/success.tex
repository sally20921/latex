\documentclass[ebook,12pt,oneside,openany]{memoir}
\usepackage[utf8x]{inputenc}
\usepackage[english]{babel}
\usepackage{url}

\begin{document}

\title{The Success Trap}
\maketitle
\section{Introduction}
How do you know when to leave your job or change your life?

There are a few different hypotheses on why people stay in jobs they don't like. Many of these ultimately come down to a materialist explanation, 
along the lines of survival; or rather, the survival of a lifestyle that people want or have to maintain.
Work is part of our human experience. We engage in creative endeavours that produce possibilities for exchange and fuel further creativity;
Ideally, these work-related endeavours should engage our playfulness as well as our resilience and ability to surmount obstacles.

What's wrong with the world of work today? While the scientific revolution and the growth of capitalism has dramatically improved our material quality of life and life expectancy, 
they have failed our deeper existential well-being.

VUCA is an acronym that stands for Volatility, Uncertainty, Complexity and Ambiguity.
It has been used to describe the challenges faced by those organizations.
Whatever the future looks like, these changes are certainly forcing us to reconsider the definition of work and what it means to be human.

Essentially, life is uncertain and we can engage with it more healthily through better self-understanding.
We can think of the success trap as a materialistic one: where having a particular lifestyle and responsibilities keep you in a job you don't like very much.
While the material aspect may be a factor, the psychological aspect is just as important-and it is here that intervention is possible to help us gain freedom from this trap.

We think that by controlling uncertainty, we will be safe. This expectation simply isn't realistic in today's world.
Refusing to make time and space to examine our relationship to the uncertain with curiosity and compassion keeps us in a controlling reflex that traps us in a less than satisfactory state of affairs.
We are attached to how things should be or want to return to how things were in the pas.

Also, you have too much to lose, so you continue with the job you're not so inspired with. This is called the sunk cost fallacy and it's what keeps you investing in a dead end because you have already invested so much.
These fast and efficient processes seem to separate us from our creativity and our ability to relate deeply with objects and people. They seem to distance us from nature, the source of everything, and the deeper social fabric.
If we are overwhelmed and scattered, it is much more difficult to feel whole. 
It's been argued that burnout is a result of repeated and unsustainable compromise of one's deeply held values-like well-being or compassion. 
If a person is in a situation where they are forced into the repeated and unsustainable compromise of their values-how they see themselves, others and the world-then what we call burnout is not an individual problem.

The environment or culture we live and work in may exacerbate this disconnection, by encouraging a 'do or die' mentality where individuals plough on and ignore their emotional, psychological and physical needs; or through a constant stream of information and interactions leading to an inability to switch off.
Even though being chased by hungry sabre-toothed tigers is no longer a primary concern in our society, our work culture and psychology still carry remnants of this quality of survivalism.

You've probably come across an environment that you would describe as toxic. You might think of an environment where competitiveness dominates. There's a constant sense of threat and manipulative politicking.
You may also feel a shortage of resources and of having to do more with less.

You have a high workload with little control over it, a lack of autonomy under stressful conditions. This has been shown to lead to `learned helplessness' and is linked to several mental and physical illnesses.
Systems that pit individuals against each other, in order to squeeze productivity gains and push people to perform harder, may be the most damaging of all.
Leaders may do much better in measuring people's work by the value they produce rather than the hours they put in.
Leaders who help their employees feel safe, valued, and appreciated are the most needed.

If bosses are sold on the profit-driven work culture, it's likely they'll reinforce these messages for you.
They might heighten your sense of insecurity so that you're less likely to entertain alternatives.
Many bosses have created atmospheres where everything is centered around them and other people's needs are ignored.

The idea that those who succeed in terms of education, occupation, and income might eventually be at a disadvantage in terms of mental health, happiness and fulfillment is not necessarily an intuitive one.
If you're a high achiever in that sense, you can easily feel guilty about your career crisis.
Rescuer syndrome is concept used in psychology to capture some of the patterns of though and behaviour that lead people to sabotage their own potential while meaning to help others.
Of course, the notion of self-sacrifice is noble. Inevitably, these identities are a risk factor for compassion fatigue, resentment and burnout.

People believe they have to be constantly productive and busy or risk losing their jobs. As workloads continue to increase and become more complex in the VUCA world, an obsession has developed with productivity tools and time management hacks. 

\section{Goal-driven versus creative flow}
When does working towards goals become unhealthy? The answer is when it becomes a compulsion rather than a choice.
We could call this phenomenon `goal addiction'. Any addiction involves the reward centers of the brain.
These are the parts of the brain that tell you what you should chase, and give you a hit of pleasurable dopamine when you get it.
The consequence is that you can burn out or waste your energy on meaningless goals that are not aligned with your deeper potential and values. 

Without this recovery time, you can never truly reach your potential since your mind doesn't have time to reflect, learn, integrate and prepare for the next cycle of activity.

I believe that the opposite of being goal-driven is not being goal-less;it is being in creative flow.
A goal-driven task is done for the sake of the goal, whereas entering a state of flow is about the activity for the activity's sake.
Rather than setting fixed goals and working towards them, you can articulate possibilities and uncover creative solutions to the obstacles that come up.
Rather than making something happen, you're removing the obstacles so that the actions unfold effortlessly.

But why aren't we more connected to this creative way of engaging with life? Why is pushing to achieve goals still the dominant cultural model of productivity?

Creativity was defined as the ability to come up with original ideas, think in a detailed and elaborative way, synthesize information, and be open-minded and curious.
General traits among people considered creative in the wider sense were identified in the 2016 study. These include:
\begin{itemize}
    \item \textbf{Spontaneity} a tendency to be flexible and act fast on new opportunities, approaching them with an open mind and a playful perspective, which can come off as impulsive.
    \item \textbf{Playfulness} light-heartedness and a drive to explore the world, which can be perceived as mischievious.
    \item \textbf{Autonomy} striving for independence in your thoughts and actions, relying on intrinsic motivation to pursue goals. Such individual can seem out of control.
    \item \textbf{Defiance} a tendency to reject existing norms and authorities in pursuit of their own ideas. This allows you to see what others cannot see and develop solutions that push boundaries, which can seem rebellious.
\end{itemize}

The best ideas seem to come up at unexpected times or in unusual situations when you're not hyper-focused on getting things done or solving a problem.
So how can we develop this creativity and access the ability more purposefully? The key ingredient is \textbf{worrying less about what others think of you, switching off your inner critic and letting go of self-anxiety}.
If you're harsh and critical with yourself as you try to come up with a solution, your creativity will be dampened. 

MRI scans show that areas of the brain that monitor for social approval are switched off while they perform and that this leads to the emergence of spontaneous creative activity.
The self is constantly changing and with the growing understanding of the brain as neuroplastic, we know that learning and moulding of our patterns of thinking and behaving is a lifelong process if we so choose.
What's interesting is that the absence of a conherent persona has been found to be a hallmark of creative genius. 
Creative geniuses are more likely to display contradictory behaviors and tendencies rather than one coherent personality.

There may also be times, places, people and habits that dampen that quality of openness to new ideas in you.
Productivity has become equated with automatioßn and efficiency rather than innovation and spontaneity.
But humans need time and space to think their best thoughts. It's usually during rest, play or time in nature that the genius idea comes through; not whne you're hyper-focused on information gathering or problem solving.
Solving complex 21st century problems require that we drop the biggest productivity myth of all: that machine-like productivity is more valuable than genius-like creativity.

The solution? It's time to slow down and make space. This may feel a bit counter-intuitive, as your instinct is to keep running on the treadmill.
In this fast-paced, urgent, VUCA world, giving yourself permission to slow down is probably the kindest and wisest thing you'll ever do for yourself.

It's unfortunate that most people wait for a crisis or burnout before paying attention to their self-care.
The healthier approach to being overwhelmed is to increase your self-care, not sacrifice it.
Optimal performance and self-care go hand in hand. 

The key principle in time management is that there's no such thing as time management: it's all choice management.
If you're doing something out of fear or obligation, you're much more likely to feel drained. 
The inner boss is fuelled by the fight-flight-freeze responses of the malmmalian-reptilian brain.

\end{document}