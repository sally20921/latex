\documentclass[ebook,12pt,oneside,openany]{memoir}
\usepackage[utf8x]{inputenc}
\usepackage[english]{babel}
\usepackage{url}

\begin{document}

\title{The Seeking System}
\maketitle
\section{Our Organizations Are Letting Us Down}

Let's start with a couple of questions.
Are you excited about your work? Or does work makes you feel like you need to ``shut off''
in order to get through it?

Employees want to feel motivated. They seek meaning from their jobs.
But some realities of organizational life are preventing them from feeling alive at work.

As the Gallup studies suggest, a majority of employees don't feel they 
can be their best selves at work. 
They don't feel they can leverage their unique skills or find a sense of purpose 
in what they do. 
Most organizations aren't tapping into their employee's full potential, 
resulting in workplace malaise and dull performance.

Here's the thing: many organizations are deactivating the part of employees'
brains called the \textbf{seeking system}. 
Our seeking systems create the natural impulse to explore our worlds, 
learn about our environments, and extract meaning from our circumstances.
When we follow the urges of our seeking system, it release dopamine-a neurotransmitter linked to 
motivation and pleasure-that makes us want to explore more.

The seeking system is part of the brain that encouraged our ancestors to explore
beyond Africa. And that pushes us to pursue hobbies until the crack of dawn and seek out 
new skills and ideas just because they interest us.
The seeking system is why animals in captivity prefer to search for their food 
rather than have it delivered to them.
When our seeking system is activated, we feel more motivated, purposeful, 
and zestful. We feel more alive.

Exploring, experimenting, learning: this is the way we're designed to live.
The problem is that our organizations weren't designed to take advantage of people's seeking systems.
Organizations were purposely designed to suppress our natural impulses to learn and explore.

Instead organizations activate their fear systems, which narrows their perception and encourages their submission.
When people work under these conditions, they become cautious, anxious, and wary.
They wish they could feel lit up and creative, but everything starts to feel like a hassle.
They start to experience depressive symptoms.
Over time, they begin to believe their current state is unchangeable, and 
they disengage from work.

This is our body's way of telling us that we were designed to do better things.
To keep exploring and learning. This is our biology-it is part of our adaptive unconscious to know
that our human potential is being wasted, that we are wasting it away.
When the seeking systems are not active, human aspirations remain frozen in an endless winter of discontent.

Things are different now. They need employees' creativity and enthusiasm in order to survive, 
adapt, and grow. They need to activate their employees' seeking systems.

\section{The Way Things Ought To Be}
With their explorable worlds and open-ended outcomes, games are catnip for our brains.
The circumstances that evoked those feelings and dynamics can be replicated in other environments, too.
I'm talking about improving people's enthusiasm toward experimentation, innovation, and learning.

Expressing her unique skills as part of a team, exploring new things,
and finding a sense of purpose-the three triggers that activate the seeking system.

\subsection{Self-Expression}
Even though she had a specific role, she was free to imagine it and interpret it 
any way she saw fit to help her team succeed.
She was able to leverage her character's unique abilities and powers.

\subsection{Experimentation}
The game also featured a continual process of exploring and learning.
Most employees also would love it if their work swept them away for four 
and a half hours-the effect of dopamine-rather than forced them to watch the clock,
wishing it would move faster.

This meant that she could afford to think up new approaches and try them out,
reacting according to how the environment responded to their previous ideas.

\section{The Benefits of the Seeking System}
Most neuroscientists agree that one of the most basic emotional systems pertains to a functionally identifiable
neural circuit that depends on dopamine, and that emotional system might be called interest, anticipation, or seeking.
This means that the seeking system is a real place in the brain:
a neural network that runs between the prefrontal cortex and the ventral striatum.

These circuits appear to be major contributors to our feelings of engagement and excitement 
as we seek the material resources needed for bodily survival, and also when we pursue
cognitive interest that brings positive existential meaning into our lives.

When we feel an urge to try new things and learn as much as possible about our environments, 
our seeking circuits are firing.
This happens when we are curious or find something unexpected or anticipate something new.
As a result, we experience a jolt of dopamine, which feels pleasurable and can be thrilling.
And since dopamine regulates our perception of time, we experience time differently, so that 
we might report that it seems to stand still even as it rushes by.
When the seeking system is activated, we experience persistent feelings of interest,
curiosity, sensation seeking, and in the presence of a sufficiently complex cortex, the search for higher meaning.

This is a positive, invigorated feeling of anticipation that results in zest. 
Zest leads people to live life with a sense of excitement, anticipation, and energy.
When we feel zestful, we see life or work as an adventure. 
And we approach new situations and changes with enthusiasm and excitement instead of apprehension and anxiety.

When we're excited and follow our body's intrinsic urge to learn new things,
the world feels like a better place to live, and we become more creative and productive.

\subsection{Performance}
As it turned out, people in the excited condition sang much better than the people 
who cued themselves to be anxious. 
Either way, physiological arousal occurs in people who have to sing in front of a stranger.
If we interpret this arousal as anxiety, the fear encumbers our enthusiasm and creativity.
When singers interpret their arousal as excitement, their seeking systems surge, causing them 
to be more playful, optimistic, and creative.

If we can trigger our seeking system during stressful experiences, it promotes 
adaptive response that helps shift negative stress states to more positive ones.

When people try to become calm under physiological arousal, they are telling themselves that the arousal is bad-that it is unwelcome.
They code the same arousal as threat and anxiety, which activates fear, shuts down creativity, and hinders problem solving.

When athletes interpret their high arousal as excitement, they are more likely to exhibit playful, learning-oriented behaviors.

This is why leaders need to know how to activate people's seeking systems.
When you increase enthusiasm and excitement, you improve problem solving and creativity.
This is how most people want to feel in their jobs-not only because these feelings lead to better 
outcomes, but because we spend most of our waking hours at work, and positive emotions put more living into life.

\subsection{Motivation}
There are other benefits to the seeking system. 
Unlike the short-term motivation of extrinsic rewards such as financial bonuses,
an activated seeking system has longer-term impacts on our motivation.

It shows that the seeking system prompts an intrinsic urge to explore, rather than
giving extrinsic reward for an action.
When the seeking system is stimulated, an animal feels the urge to explore and investigate,
to find whatever is potentially useful in the environment.

The seeking system doesn't seem to reward us for innovation and creativity, 
but rather it drives and propels those behaviors.
The distinction is mainly temporal-reward happens after a behavioral event, whereas seeking happens before it.

The mammalian brain has separate systems for what he calls wanting and liking.
Wanting is Berridge's terminology for the seeking system, whereas the liking system is the brain's reward center.
When we experience the pleasure of a reward, it is the opioid system, rather than the dopamine system, that is being stimulated.
These systems lead to very different effects: dopamine has an animating effect; opiates induce a happy stupor.

This animating effect of the seeking system is optimal in work settings because it urges us into action
instead of making us complacent. They wanted to explore and learn more and more.

Seeking systems are not placated after we've achieved a goal. In this sense, 
when the seeking system is paired up with a complex cortex, it is related to Maslow's ideas about self-actualization:
even if all these needs are satisfied, we may still often expect that a new discontent and restlessness will soon develop.
This need we may call self-actualization, the desire to become more and more what one is, to become everything that one is capable 
of becoming.

So our seeking circuitry just won't rest, even when we have acquired the material possessions we lusted for.
Even after we received lots of extrinsic rewards and all our needs are fulfilled, our seeking systems still push us to find the best way to use our unique skills-and then do it.

This is the way we're meant to live. It's our biological imperative.
Through evolution, we've retained our emotional impulses to explore, experiment, and learn.
Part of our brain urges us to learn new things and find new ways to use our unique skills, instead of performing monotonous generic tasks.
And when we follow these urges of the seeking system, we get a dopamine release that not only feels good, it motivates us to explore more.

\subsection{Happiness and Health}
When we're working and living in environments where exploration and experimentation are encouraged, we're happier people.

Empty positive emotions are about as good for you as adversity.

This is where our seeking systems fit in. That circuit in our brains provides an intrinsic impulse for us 
to keep exploring our environments, learning, and finding meaning.

The seeking system offers a biological way to understand and predict exciting and purposeful activities and feelings
outside the realm that is normally thought of as rewarding in a pleasurable or hedonic sense.

Maybe, this is what the seeking system is urging us to do: to explore our environments in order to discover our personal potential
in the world, then express ourselves in that way.
Following our seeking system's urges make us feel good in a purposeful way, which makes us healthier and happier.

\section{The Way Things Are-And How To Make Them Better}
Those experienced dogs had learned helplessness. They had learned to give up and passively accept the shock 
since there was nothing they could do about it. 
They learned to take it because their past experiences taught them that hey were helpless.

Learned helplessness is a three-headed monster. It alters our emotional states, lowers our motivation,
and changes our cognitive reasoning.
Learned helplessness often continues when we go from one situation to another, and it is reliably difficult 
to overcome once it sets in. 

Many employees find themselves caught in a crossfire between their biological seeking systems and their organizational realities.
Employees begin to ignore the urges of their seeking systems. This means they shut off the dopamine and let their anxieties dominate.
Many employees learn how to shut off and just take it-an attitude that results in disengagement at work and depressive symptoms in life.
They end up making a living but not a life.

\subsection{Fear is Kryptonite to the Seeking System}
The seeking system is the motivational engine that each day gets mammals to venture out into the world,
even though they don't know what they'll find. 

But say there was a battle between our seeking systems and our fear systems.
Which one would win?

For evolutionary reasons, fear will win. Play is how we learn what we're capable of.
What we are seeing here is the inhibiting relationship between the seeking system and the fear system.
When one system is activated, the other shrinks back.

Playfulness is inhibited by negative emotions such as fear.

Fear-based management solved the problem of how to control so many employees when you couldn't develop trust.
This became the root of employees' learned helplessness at work.

The seeking system is built for this proactive approach, because it creates enthusiasm and curiosity.
Dopamine doesn't just feel good, it makes employees into a volunteer army that begins change rather than resisting it.

The bad effects of industrialization on the seeking system are still lingering.

Fine-tuned control that exploits existing processes makes it hard for workers to explore and experiment with new alternatives.

\subsection{Balancing the Freedom and the Frame}
What we need to do is help employees find freedom in that frame.
The freedom refers to the space where employees can experiment, try new things,
express themselves, and play to their strength.

\section{Self-Expression}
\subsection{Encouraging People to Bring their Best Selves to Work}
With all this ambiguity going on, this also is the time we experience the basic human need to fit in and 
be accepted by our new managers and coworkers. 
It is a time of anxiety because it's a vulnerable situation to be in.

When newcomers write about and share stories about their best selves with others,
it leads to greater performance and retention. 
As a consequence, they were viewed by their coworkers in the way they wanted to be seen. 
To put it another way, they felt like their best selves. 

A best self is the cognitive representation of the qualities and characteristics the individual displays when at his or her best.
They comprise the skills and traits that we've developed and discovered over time, and the actions we have taken to affect others in a positive way.

Because they could express themselves more authentically. 
People were proud to be recognized as individuals. 
This gave them a distinctive identity within the organization and helped them identify 
with the organizations much faster. 

When people are prompted to think about their best traits, their seeking systems are activated.
It turns us on to think about what we are capable of, and this is how our seeking systems help us bring more energy and engage more of ourselves at work.
\end{document}