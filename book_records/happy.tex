\documentclass[ebook,12pt,oneside,openany]{memoir}
\usepackage[utf8x]{inputenc}
\usepackage[english]{babel}
\usepackage{url}

\begin{document}

\title{What Makes Your Brain Happy and Why You Should Do the Opposite}
\maketitle

Years of neuroscience research have led to the current understanding of the brain as a prediction machine-an amazingly complex organ that processes information to determine what's coming next.
Specifically, the brain specializes in pattern detection and recognition, anticipation of threats, and narrative.
The brain lives on a preferred diet of stability, certainty, and consistency.
And it perceives unpredictability, uncertainty, and instability as threats to its survival.

The problem is that our brains' evolved capacity for avoiding and defending against these threats.

What I wish to communicate with the metaphor of a happy brain is simply that under various conditions, our brains will tend toward a default position that places greatest value on avoiding loss, lessening risk, and averting harm.
These protective tendencies can go too far and become obstacles instead of virtues.

\section{Certainty and the Seduction of Chance}
The human brain is the undisputed learning master on the planet. 
Neuroscience research is revealing that the state of not being certain is an extremely uncomfortable place for our brains to live:
the greater the uncertainty, the worse the discomfort. Even a small amount of ambiguity triggers increased activity in the amygdalae-two deep brainstructures that play a major role in our response to threats.
At the same time, the brain shows less activity in the ventral striatum, a part of the brain involved in our response to rewards.
As level of the ambiguity increases, amygdalae activity continues to increase, and ventral striatum activity continues to decrease.

The same parts of the brain that respond to physical threats are also the parts that respond to belief-based threats.
What this tells us is that the brain doesn't merely prefer certainty over ambiguity-it craves it.

Ambiguity, which might result from considering the new information, is a threat.
The brain wants the same things: stability and consistency. But very nearly everything we do is colored by this drive.

People with less need for ``cognitive closure'' were typically more creative problem solvers than their counterparts.

\end{document}