\documentclass[ebook,12pt,oneside,openany]{memoir}
\usepackage[utf8x]{inputenc}
\usepackage[english]{babel}
\usepackage{url}

\begin{document}

\title{Why Motivating People Doesn't Work}
\maketitle

\section{The Motivation Dilemma}
The motivation dilemma is that leaders are being held accountable to do something
they cannot do-motivate others.

Ultimately, how you feel about the meeting has the greatest influence on your sense of 
well-being. Your well-being determines your intentions, which ultimately lead to your behavior.

Every day, your employees' appraisal of their workplace leaves them with or without a positive sense of well-being.
Their well-being determines their intentions, and intentions are the greatest predictors of behavior.

The appraisal process is at the heart of employee engagement-and disengagement.
Researchers have only recently explored how people come to be engaged.

Motivating people may not work, but you can help facilitate people's appraisal process so they are more likely to experience
day-to-day optimal motivation.
Optimal motivation means having the positive energy, vitality, and sense of well-being required to sustain the pursuit and achievement of meaningful goals 
while thriving and flourishing.

Motivation is a skill. People can learn to choose and create optimal motivational experiences anytime and anywhere.

Asking why people were motivated to attend the meeting leads to a spectrum of motivation possibilities represented as six motivational outlooks.
\begin{itemize}
    \item \textit{Disinterested motivational outlook}-You simply could not find any value in the meeting; if felt like a waste of time, adding to your sense of feeling overwhelmed.
    \item \textit{External motivational outlook}-The meeting provided an opportunity for you to exert your position or power; it enabled you to take advantage of a promise for more money or an enhanced status or image in the eyes of the others.
    \item \textit{Imposed motivational outlook}-You felt pressured because everyone else was attending and expected the same from you; you were avoiding feelings of guilt, shame, or fear from not participating.
    \item \textit{Aligned motivational outlook}-You were able to link the meeting to a significant value, such as learning-what you might learn or what others might learn from you.
    \item \textit{Integrated motivational outlook}-You were able to link the meeting to a life or work purpose, such as giving voice to an important issue in the meeting.
    \item \textit{Inherent motivational outlook}-You simply enjoy meetings and thought it would be fun.
\end{itemize}

The three suboptimal motivational outlooks-disinterested, external, and imposed-are the junk foods of motivation.
Their tangible or intangible rewards can be enticing in the moment, but they do not lead to flourishing.
Even if they have the energy it takes to achieve their goals, they are not likely to experience the positive energy, vitality, or sense of well-being required to sustain their performance over time.

The three optimal motivational outlooks-aligned, integrated, and inherent-are the health foods of motivation. They might require more thought and preparation, but they generate high-quality energy, vitality, and positive well-being that leads to sustainable results.

Motivating people does not work because they are already motivated-they are always motivated. The motivation dilemma is that even though motivating people doesn't work,
leaders are held accountable for doing it. This dilemma has led to ineffective motivational leadership practices. You push for results, only to discover that pressure, tension, and external drives prevent people from attaining those results.
Traditional motivational tactics focus on obtaining short-term results that tend to destroy long-term prospects.

Human beings have an innate tendency and desire to thrive. We want to grow, develop, and be fully functioning.
Striving to reach our individual potential is natural, yet we innately recognize that the inter-connection between ourselves and the world around us is a vital part of that process.
Human thriving in the workplace is a dynamic potential that requires nurturing. 

What does work? The essence of the answer lies at the heart of the science of motivation and the revelation of three psychological needs-autonomy, relatedness, and competence.

Autonomy is our human need to perceive we have choices. It is our need to feel that what we are doing is of our own volition. It is our perception that we are the source of our actions.
Diverse studies over the past twenty years indicate that adults never lose their psychological need for autonomy. 
Autonomy doesn't mean that managers are permissive or hands-off but rather that employees feel they have influence in the workplace.
If people don't have a sense of empowerment, their sense of autonomy suffers and so do their productivity and performance.

In light of what we know about autonomy, quiet coaches get better results than verbal coaches do because the verbal encouragement externalizes the exercisers' attention and energy.
The external encouragement and praising subverted the coaching subjects' own internal desire to perform, push and excel-and thus limited their capacity to do so.

Relatedness is our need to care about and be cared about by others. It is our need to feel connected to others without concerns about ulterior motives.
It is our need to feel that we are contributing to something greater than ourselves.
Notice the range of needs that relatedness covers. It is personal, interpersonal, and social. We thrive on connection.
Beliefs such as ``It's not personal; it's just business'' diminish an aspect of work that is essential to our healthy functioning as human beings-the quality of our relationships.
When managers apply pressure to perform without regard to how it makes people feel, people interpret the managers' actions as self-serving.
The role you play as a leader is helping people experience relatedness at work: caring about and feeling cared about, feeling connected without ulterior motives, and contributing to something greater than oneself.

As a leader, you can encourage relatedness by challenging beliefs and practices that undermine people's relatedness at work. That means paying attention to how your people feel.
That means gaining the skill to deal with their emotions. That means \textbf{getting personal}.

Competence is our need to feel effective at meeting everyday challenges and opportunities. It is demonstrating skill over time. It is feeling a sense of growth and flourishing.
Motivating people doesn't work because you can't impose growth and learning on a person. But you can promote a learning environment that doesn't undermine your people's sense of competence.

If her needs for autonomy, relatedness, and competence are satisfied, the result is an aligned, integrated, or inherent motivational outlook. 
Imagine you have a manager with control needs. She micromanages people and projects-whether they need it or not.
Her micromanagement is undermining your sense of autonomy-she is controlling your work and not allowing you to think for yourself.
Your manager's lack of sensitivity to your needs, and apparent self-interest prevent any sense of relatedness.
You have an imposed motivational outlook. Driven by fear and maybe a little guilt, you do your job. 

\section{The Danger of Drive}
A person with a lot of drive is considered to have a lot of motivation. 
Your psychological needs are not drives. In fact, they are just opposite. Drive dissipate when they are satiated.
However, when psychological needs are satisfied, you experience such positive energy, vitality and a sense of well-being that you want more.
You have probably experienced this with your own positive addictions. 

Brandt began to acknowledge that something external was driving him and prompting his emotions, feelings, and actions.
That something was his need to prove himself, fueled by a desire to impress his father.
The revelation of why our sense of well-being, intentions, and behavior are dysfunctional is groundbreaking.
It turns out that Brandt had been bouncing back and forth for years between a craving for praise, validation, and tangible rewards, and a fear of letting his father down.
Ironically, his desire to please his father prevented him from experiencing a true and authentic sense of relatedness.

Self-regulation is mindfully managing feelings, thoughts, values, and purpose for immediate and sustained positive effort.
An individual is much more likely to have high-quality self-regulation in a reliable, safe, and trusting work setting.

The danger of drive is that it promotes external motivators that undermine people's psychological needs for autonomy, relatedness and competence.
External motivational drivers comes in intangible forms such as approval, status, shame or fear.
When employees focus on an external motivator, they lose autonomy. People ultimately resent leaders who create a pressurized workplace that undermines autonomy.
Moreover, people regard managers who drive for results as self-serving.
Conditional support undermines people's relatedness.
Driving for results by adding pressure and tension blocks people's creativity and ability to focus, leaving them feeling inadequate or ineffective at coping with circumstances-which undermines their competence.

\section{Motivation is a Skill}
\begin{itemize}
    \item Identify your current motivational outlook.
    \item Shift to an optimal motivational outlook.
    \item Reflect by noticing the difference between having a suboptimal motivational outlook and having an optimal motivational outlook.
\end{itemize}

Teaching leaders about motivation is difficult because they believe their job is to motivate others-not themselves.
Doing tasks or pursuing goals because we think we have to or to avoid negative emotions of guilt, shame, and fear.
Relatedness is about mutually caring relationships without ulterior motives; it is pure and free of pressure, stress, or obligation to prove you care.
Relatedness is not about appeasing someone, keeping the peace with her, or worrying that she might retreat from the relationship if you don't live up to her expectations.

\end{document}