\documentclass[ebook,12pt,oneside,openany]{memoir}
\usepackage[utf8x]{inputenc}
\usepackage[english]{babel}
\usepackage{url}

\begin{document}

\title{Choice Theory}
\maketitle

Besides survival, I believe we are genetically programmed to try to satisfy four psychological needs: love and belonging, power, freedom and fun.
Evolution has provided humans and higher-order animals with genes that grant us the ability to feel.
We struggle to feel as good as we can and as much as we are able, try to avoid feeling bad.
How we feel tells us with great accuracy how well we are satisfying our need for love and belonging.
This book focuses on social activity: how giving up external control can help us to get along better with each other.

We make and keep friends easily. It is love, mainly sexual love, that is most frustrating part of this need.
Good friends can keep their friendship going for a lifetime because they do not indulge in the fantasies of ownership.
When one use external control, the family is often torn apart.

In the external control society we live in, the powerful often define reality, even though this definition may be harmful to others.
But at a minimum, we want someone to listen to what we have to say. If no one listens to us, we feel the pain of the powerless.
The powerful would find that there is more power in getting along with people than in trying to dominate them.

External control, the child of power, is the enemy of freedom. Whenever we lose freedom, we reduce or lose what may be a defining human characteristic:
our ability to be constructively creative. When we don't feel free to express ourselves, or if we do and no one will listen to us, our creativity may cause us pain or even make us sick.

Fun is the genetic reward for learning. We are descended from people who learned more or better than others.
We are the only creatures who play all our lives, and because we do, we learn all our lives. 

It takes a lot of effort to get along well with each other, and the best way to begin to do so is to have some fun learning together.

How can I figure out how to be free to live my life the way I want to live it and still get along well with the people I need?
Power destroys love. No one wants to be dominated, no matter how much those who dominate protest their love. 
By now it is obvious that we are social beings, and to satisfy our needs we must have good relationships. Misery is being without the people we want and need.

Any time we feel very good, we are choosing to behave so that someone, something, or some belief in the real world has come close to matching a picture of that person, thing, or belief in our quality worlds.
Anytime we are able to succeed in satisfying a picture in this world, it is enjoyable. 
As we attempt to satisfy our needs, we are continually creating and re-creating our quality worlds.
We all need happy, supportive people in our quality worlds; nothing less will do. 

\textbf{Too many teachers and bosses do not realize how much they are needed just to be warm, friendly, and supportive to those they manage}.
But many who teach and manage don't understand that given care and support, the students and workers who are doing so little now would be willing and eager to work hard.
Without sufficiently supportive people in our quality worlds, we often follow an extreme version of the fourth variation of unhappiness: we try to force ourselves to do what goes against our basic needs.
To get along better than we do now with another person, we need to try to learn what is in that person's quality world and then try to support it.
Doing so will bring us closer to that person than anything else we can do.
The best thing to do if you know choice theory is to explain the quality world and what you are afraid of to your partner.
It is common for people that it is your right to make people do what you want them to do-to put a picture in their quality worlds that goes beyond relating, to actually owning someone.
If you own that person, it is right to make him or her do what you want.
Any ownership picture is a relationship disaster in the making. 

It is especially hard for powerful people to be tolerant of the quality worlds of people who are less powerful. 
If everyone could learn that what is right for me does not make it right for anyone else, the world would be a much happier place.
Choice theory teaches that my quality world is the core of my life; it is not the core of anyone else's life. This is a difficult lesson for external control people to learn.

When we depress, we believe we are the victims of a feeling over which we have no control.
All behaviors that have anything directly to do with satisfying basic needs are chosen.
We are always trying to choose to behave ina way that gives us the most effective control over our lives.
In terms of choice theory, having effective control means being able to behave in a way that reasonable satisfies the pictures in our quality worlds.
It was clear in the therapy that Todd had the ability to make better choices even when he was strongly choosing to depress.

\section{Choice Theory in the Workplace}
Boss management, my designation for external control psychology in the workplace, is still very much the norm.
Although many school administrators boss teachers and do untold harm to education, they have nowhere near the power that most private-sector bosses have to send the message: I am someone to fear.
But if high-quality work is what the manager is trying to achieve, fear is the worst strategy. 

The longer a lower-level manager has been exposed to bossing, the more he or she uses it, even the senior worker tends to boss the other.
The specific harm of boss management is that it prevents anyone who is bossed, from putting the poeple above them into their quality worlds.

When the workers resist the boss, as they almost always do in a variety of ways that compromise quality, the boss uses threats and punishment to try to make them do what he or she wants.
The boss thinks this adversarial relationship is the way things should be; cooperation with workers is a subversive idea.

It depends on restraining something that has never in human history been restrained very long, something endemic to business and politics that has destroyed prosperity in every modern society the world has ever known: human greed.

It may be that love and belonging behaviors, which might be strong enough to moderate their greed in a choice theory society, are difficult to express in our external control world.
They tend to distrust. Once you give up external control psychology for choice theory, it is almost impossible for you to come into contact with people who work for you and not think about how much better it would be if we all got along well.
If these contacts are satisfying, it feels good, and you will tend to want even better relationships; that's how our genes work.

What makes boss management so destructive is that it focuses on individuals and pits them against each other. What makes lead management so successful is that it focuses on creating a cooperative system and on the belief that if you treat people well and explain what you want them to do, you can trust them to do a good job.
In the following four elements of lead management, you will continually see that the message, we care about you, is central to this effort.
Lead managers know that caring costs nothing and has a huge return. 
Lead managers keep asking themselves the core choice theory question: If I do this, will I get closer to the people who work for me or further away?
If the answer is further away, they don't do it.

Lead managers know that the core of quality is managing workers so they put the manager and the work into their quality worlds.
That is, all who are involved must get close and stay close. As in every other area discussed in this book, good relationships are the key  in the workplace.

Lead managers engage all workers in an ongoing honest discussion of both the quality and the cost of the work that is needed for the company to be successful.
They not only listen but continually encourage the workers to offer them any suggestions that will improve quality and lower costs.

The workers are responsible for inspecting their own work with the understanding that they know bets what high-quality work is and how to produce it at the lowest possible cost.
But the manager makes it clear that quality takes precedence over cost. When the workers are given this assurance, quality goes up and costs go down.
High quality depends on a level of trust between workers and managers that cannot be achieved by bossing.

The more workers are bossed, or in many instances even when they are not bossed but are so used to being bossed that they perceive every request as bossing, 
the more they enjoy using what little power they may have to obstruct.
Playing it safe and enjoying it while the company grinds to a halt is the goal of obstruction.
No is always safe, and they use it a lot.

The better the relationship the workers have with their managers, the less these complaints will surface or persist, but good relationships will not prevent all injuries.
The present way to deal with these complaints is adversarial. But as long as external control is the psychology of the workplace, it is all we have.

No person with some power should ever make a formal evaluation of a subordinate.
Only unthinking bosses like to do evaluations; it gives them the feeling of power that means so much to them, especially since they can disguise their real motives under the pretense: I'm trying to help you.
What the workers do is protect themselves as much as possible, no matter what effect it has on the company.
This procedure perpetuates a climate of distrust in the workplace.
It is one of the few things we routinely do in industry that is completely ineffective.

What is needed instead is for the company to provide a yearly opportunity for each employee to talk to his or her manager about how they both could do something to improve the company.
Instead of performance evaluations, these yearly discussions might be called company solving circles, the workplace equivalent of marriage and family solving circles.



\end{document}