

\documentclass[twoside,twocolumn]{article}

\usepackage{blindtext} % Package to generate dummy text throughout this template 

\usepackage[sc]{mathpazo} % Use the Palatino font
\usepackage[T1]{fontenc} % Use 8-bit encoding that has 256 glyphs
\linespread{1.05} % Line spacing - Palatino needs more space between lines
\usepackage{microtype} % Slightly tweak font spacing for aesthetics

\usepackage[english]{babel} % Language hyphenation and typographical rules

\usepackage[hmarginratio=1:1,top=32mm,columnsep=20pt]{geometry} % Document margins
\usepackage[hang, small,labelfont=bf,up,textfont=it,up]{caption} % Custom captions under/above floats in tables or figures
\usepackage{booktabs} % Horizontal rules in tables

\usepackage{lettrine} % The lettrine is the first enlarged letter at the beginning of the text

\usepackage{enumitem} % Customized lists
\setlist[itemize]{noitemsep} % Make itemize lists more compact

\usepackage{abstract} % Allows abstract customization
\renewcommand{\abstractnamefont}{\normalfont\bfseries} % Set the "Abstract" text to bold
\renewcommand{\abstracttextfont}{\normalfont\small\itshape} % Set the abstract itself to small italic text

\usepackage{titlesec} % Allows customization of titles
\renewcommand\thesection{\Roman{section}} % Roman numerals for the sections
\renewcommand\thesubsection{\roman{subsection}} % roman numerals for subsections
\titleformat{\section}[block]{\large\scshape\centering}{\thesection.}{1em}{} % Change the look of the section titles
\titleformat{\subsection}[block]{\large}{\thesubsection.}{1em}{} % Change the look of the section titles

\usepackage{fancyhdr} % Headers and footers
\pagestyle{fancy} % All pages have headers and footers
\fancyhead{} % Blank out the default header
\fancyfoot{} % Blank out the default footer
\fancyhead[C]{Article by Seri $\bullet$ January 2021 $\bullet$ Vol. XXI, No. 1} % Custom header text
\fancyfoot[RO,LE]{\thepage} % Custom footer text

\usepackage{titling} % Customizing the title section

\usepackage{hyperref} % For hyperlinks in the PDF

%----------------------------------------------------------------------------------------
%	TITLE SECTION
%----------------------------------------------------------------------------------------

\setlength{\droptitle}{-4\baselineskip} % Move the title up

\pretitle{\begin{center}\Huge\bfseries} % Article title formatting
\posttitle{\end{center}} % Article title closing formatting


\title{Single-Source Deep Unsupervised Visual Domain Adaptation} % Article title


\author{%
\textsc{Seri Lee}\thanks{A thank you or further information} \\[1ex] % Your name
\normalsize Seoul National University \\ % Your institution
\normalsize \href{mailto:sally20921@snu.ac.kr}{sally20921@snu.ac.kr} % Your email address
}
\date{\today} % Leave empty to omit a date
\renewcommand{\maketitlehookd}{%

\begin{abstract}
\noindent 
\end{abstract}
}

%----------------------------------------------------------------------------------------

\begin{document}

% Print the title
\maketitle

%----------------------------------------------------------------------------------------
%	ARTICLE CONTENTS
%----------------------------------------------------------------------------------------

\section{Introduction}
\cite{b1}.

\section{Adversarial Discriminative Models}
Adversarial discriminative models usually employ an adversarial objective with respect to a domain discriminator to encourage domain confusion.
In the early-stage of adversarial discriminative models, the domain adversarial training of neural networks is proposed to learn domain invariant and task discriminative representations.
It is directly derived from the seminal theoretical works of Ben et al. and directly optimizes the $\mathcal{H}$-divergence between source and target.
By deriving the generalization bound on the target risk and obtaining an empirical formulation of the objective,
Ganin et al. proposed the Domain-Adversarial Neural Network (DANN) algorithm. From this point of view, the adversarial discriminative models are originally similar to the discrepancy-based models.
Recently, a couple of adversarial discriminative models were proposed with different alogorithms and network architectures, thus differing from the discrepancy-based methods.


\bibliography{refs}
\bibliographystyle{plain}

%----------------------------------------------------------------------------------------

\end{document}