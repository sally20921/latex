\documentclass[conference]{IEEEtran}
\IEEEoverridecommandlockouts
% The preceding line is only needed to identify funding in the first footnote. If that is unneeded, please comment it out.
\usepackage{cite}
\usepackage{amsmath,amssymb,amsfonts}
\usepackage{algorithmic}
\usepackage{graphicx}
\usepackage{textcomp}
\usepackage{xcolor}
\def\BibTeX{{\rm B\kern-.05em{\sc i\kern-.025em b}\kern-.08em
    T\kern-.1667em\lower.7ex\hbox{E}\kern-.125emX}}
\begin{document}

\title{Single-Source Deep Unsupervised Visual Domain Adaptation}

\author{\IEEEauthorblockN{Seri Lee}
\IEEEauthorblockA{\textit{Computer Science and Engineering} \\
\textit{Seoul National University}\\
Seoul, Republic of Korea\\
sally20921@snu.ac.kr}
ß
}

\maketitle

\begin{abstract}
\end{abstract}

\begin{IEEEkeywords}
Domain adaptation, discrepancy-based methods, adversarial learning,
self-supervised learning, transfer learning 
\end{IEEEkeywords}

\section{Introduction}
\subsection{Domain Adaptation in context of other sample-efficient 
learning methods}
Domain adaptation techniques were introduced to address the domain shift
between source and target domains \cite{b1} and for this reason,
they have recently attracted significant interest in both academia
and industry. 
\textit{Domain adaptation} (DA), also known as \textit{domain transfer},
is a specialized form of transfer learning that aims to learn a
model from a labeled source domain that can generalize well to a 
different (but related) unlabeled or sparsely labeled target domain.
It belongs to the sample-efficient learning class, together with zero-shot
learning, few-shot learning, and self-supervised learning.
We briefly compare domain adaptation with other methods in this category.
While the unsupervised domain adaptation (UDA) does not require the annoatations 
of target  data, it usually needs a sufficient number of unlabeled target  samples 
to train the model.
Compared to UDA, zero-shot learning does not need either the target data 
annotations or the unlabeled target samples. 
However, existing methods often require some auxiliary information, 
such as the attributes of the images, or the description of the classes.
Further, zero-shot learning is trained on known classes and tested on unknown classes,
which demands the model to generalize from known classes to unknown classes.
Since the known classes and the unknown classes are from different distributions,
there is no concept of domain shift in zero-shot learning.
Different from zero-shot learning, DA deals with the same learning tasks on different
domains. 
Taking image classification as an example, both source data and target data 
have the same classes. 
Few-shot learning shares similar setting with zero-shot learning.
The difference is that few-shot learning has a few annotated samples for the unknown classes.
Few shot learning and zero-shot learning can also be grouped as low-shot learning.

Self-supervised learning (SSL) is a learning paradigm that captures
the intrinsic patterns and properties of input data without using human-provided labels.
The basic idea of SSL is to construct some auxiliary tasks solely based on the data itself
without using human-annotated labels and force the network to learn meaningful representations
by performing the auxiliary tasks well.
Typical self-supervised learning approaches generally involve two aspects:
constructing auxiliary tasks and defining loss functions.
The auxiliary tasks are designed to encourage the model to learn meaningful 
representations of input data.
The loss functions are defined to measure the difference between a model's prediction
and a fixed target, the similarities of sample pairs in a representation space (\textit{e.g.} contrastive loss),
or the difference between probability distributions (\textit{e.g.} adversarial loss).
Compared with domain adaptation, SSL does not specifically address the domain shift problem 
between different domains.

\bibliography{refs_sub}
\bibliographystyle{plain}


\end{document}