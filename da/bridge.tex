\documentclass[conference]{IEEEtran}
\IEEEoverridecommandlockouts
% The preceding line is only needed to identify funding in the first footnote. If that is unneeded, please comment it out.
\usepackage{cite}
\usepackage{amsmath,amssymb,amsfonts}
\usepackage{algorithmic}
\usepackage{graphicx}
\usepackage{textcomp}
\usepackage{xcolor}
\def\BibTeX{{\rm B\kern-.05em{\sc i\kern-.025em b}\kern-.08em
    T\kern-.1667em\lower.7ex\hbox{E}\kern-.125emX}}
\begin{document}

\title{Adversarial Domain Adaptation with Domain Mixup}

\author{\IEEEauthorblockN{Seri Lee}
\IEEEauthorblockA{\textit{Computer Science and Engineering} \\
\textit{Seoul National University}\\
Seoul, Republic of Korea\\
sally20921@snu.ac.kr}
ß
}

\maketitle

\begin{abstract}
Recent works on domain adaptation reveal the effectiveness of 
adversarial learning on filling the discrepancy between source and 
target domains. 
However, two common limitations exist in current adversarial-learning-based
methods. 
First, samples from two domains alone are not sufficient to ensure domain-invariance
at most part of latent space.
Second, the domain discriminator involved in these methods can only 
judge real or fake with the guidance of hard label, while it is more 
reasonable to use soft scores to evaluate the generated images or features,
\textit{i.e.}, to fully utilize the inter-domain information.
In this paper, we present adversarial domain adaptation with 
\textbf{domain mixup} (DM-ADA), which guarantees domain-invariance 
in a more continuous latent space and guides the domain discriminator
in judging samples' difference relative to source and target domains.
Domain mixup is jointly conducted on pixel and feature level 
to improve the robustness of models. 
Extensive experiments prove that the proposed approach can achieve 
superior performance on tasks with various degrees of domain shift 
and data complexity.
\end{abstract}

\begin{IEEEkeywords}
Domain Adaptation, semantic segmentation, object recognition
\end{IEEEkeywords}

\section{Introduction}
\cite{b1}.

\bibliography{refs_mixup}
\bibliographystyle{plain}


\end{document}